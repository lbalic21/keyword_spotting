\chapter{Zaključak}
\label{pog:zakljucak}

Sustav za prepoznavanje govornih naredbi u stvarnom vremenu uspješno je implementiran
na mikrokontrolerskoj platformi ESP32 Lyrat. Nudi jednostavniju modularnu
stukturu od postojećih (\cite{arm_kws}, \cite{tflmicrospeech}) uz sposobnost pouzdanog 
prepoznavanja većeg skupa
naredbi. Implementirani sustav može prepoznati pet
različitih naredbi te aktivirati prikladni posao dodijeljen svakoj naredbi. 
Također, omogućen je odabir 
drugačijeg podskupa govornih naredbi, a uz malo više truda moguće je proširiti skup naredbi
željenim zvučnim zapisima koji će predstavljati naredbu prilagođenu korisniku.
Osim toga, modularna struktura sustava
omogućuje laganu prilagodnu pojedinih dijelova za različite namjene. Na taj način stvorena je
osnova za implementaciju različitih sustava za prepoznavanje drugih vrsta senzorskih podataka
ili čak korištenje drugih vrsta neuronskih mreža. Ono što bi moglo poboljšati sustav
u budučnosti je
nekorištenje brojeva s pomičnim zarezom u generiranju MFC koeficijenata. Na konkretnom
mikontroleru to ne predstavlja prevelik problem zbog jedinice
za operacije s pomičnim zarezom (engl. \textit{Floating Point Unit ili FPU}),
ali na drugim mikrokontrolerima bi takav pristup osjetno ubrzao cijeli proces.


