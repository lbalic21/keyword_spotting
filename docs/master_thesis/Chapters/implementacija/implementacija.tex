\chapter{Implementacija i rezultati}
\label{pog:implement}

Informacije o strukturi i parametrima modela neuronske mreže spremljeni su u obliku polja
programskog jezika C. Sustav na mikrokontroleru polje učitava i koristi na način detaljno objašnjen 
u poglavlju o aktivaciji neuronske mreže na mikrokontroleru \ref{sec:activation}.
Kako bi cjelokupni sustav radio u skladu sa zahtjevima, potrebno je u aplikaciji
na mikrokontroleru dodati identične naredbe koje su odabrane prilikom treniranja
modela. Stvaranje naredbi te njihovo dodavanje objektu zaduženom za prepoznavanje
naredbi prikazano je u isječku koda \ref{code:commands}. Kategorije "pozadina" (engl. "background") i
"nepoznato" (engl. "unknown") su predstavljene objektima klase BlankCommand, dok su ostale naredbe 
instance klase PrintCommand. Rezultat toga je ispis imena naredbe na konzolu u slučaju
prepoznavanja govorne naredbe. Detalj koji je ključan za ispravan rad sustava je redoslijed
dodavanja naredbi. On mora odgovarati redoslijedu koji je određen pri treniranju modela koji
je pak određen redoslijedom učitavanja podataka za treniranje u Jupyter bilježnicu. Nakon
preuzimanja podataka s interneta, mape s podacima su poredane abecedno tako da 
će i krajnji redoslijed naredbi biti takav.

\begin{lstlisting}[language=C++, caption=Stvaranje naredbi, label=code:commands]
BlankCommand command_back("BACKGROUND", 1, 0.7);
PrintCommand command_left("LEFT", 5, 0.8);
PrintCommand command_no("NO", 3, 0.80);
PrintCommand command_right("RIGHT", 3, 0.85);
BlankCommand command_unknown("UNKNOWN", 1, 0.7);
PrintCommand command_yes("YES", 5, 0.85);
PrintCommand command_zero("ZERO", 3, 0.85);

recognizer.addCommand(&command_back);
recognizer.addCommand(&command_left);
recognizer.addCommand(&command_no);
recognizer.addCommand(&command_right);
recognizer.addCommand(&command_unknown);
recognizer.addCommand(&command_yes);
recognizer.addCommand(&command_zero);
\end{lstlisting}

Zbog nesavršenosti skupa na kojem je treniran model i nejednakosti kvalitete
zvučnih snimaka, neće svaka naredba biti prepoznata na isti način. Prvo, razlikovat će
se pouzdanost vjerojatnosne intepretacije za različite klase. Drugo, uslijed izgovorene
naredbe neke će klase imati najveću vjerojatnost različit broj iteracija rada sustava.
To se događa zbog toga što sustav neprestano vrsti glavnu petlju opisanu u poglavlju
o strukturi sustava \ref{pog:struktura_sustava} i trebao bi odraditi nekoliko iteracija tijekom izgovora
jedne naredbe. Zbog svega navedenog potrebo je kalibrirati svaku naredbu zasebno.
Kalibracija se radi promjenom parametara pri konstrukciji objekta također prikazanog 
u isječku koda \ref{code:commands}. Značenje pojedinog parametra detaljno je opisano u 
\ref{sec:prepoy}. 

Još jedna stvar koju je potrebno eksperimentalno utvrditi jest sposobnost sustava da
odradi sve potrebne zadatke na vrijeme. Ne smije se dogoditi propuštanje akviziranja
novih podataka zbog kašnjenja bilo kojeg drugog dijela sustava jer se time narušava
svrha rada cjelokupnog sustava. U tablici \ref{tab:rts} prikazana su prosječna
vremena potrebna za odrađivanje poslova pojedinih dijelova sustava. 

\begin{table}[htb]
    \centering
    \begin{tabular}{|l|r|}
        \hline
        \textbf{Dio sustava} & \textbf{Vrijeme(ms)} \\ \hline
        Akvizicija uzoraka & 0.016\\ \hline
        Generiranje značajki & 4.9 \\ \hline
        Aktivacija neuronske mreže & 77 \\ \hline
        Prepoznavanje naredbi i aktivacija posla & 0.028 \\ \hline
    \end{tabular}
    \caption{Vrijeme potrebno za određeni posao}
    \label{tab:rts}
\end{table}

Iz prikazanih podataka
vidljivo je da je vremenski najzahtjevniji posao aktivacija neuronske mreže, slijedi ga 
generiranje značajki, a akvizicija i prepoznavanje naredbi imaju trajanje zanemarivo u odnosu
na prethodna dva. Svaka nova iteracija uzima STEP\_SIZE novih uzoraka na obradu. U ovom slučaju
taj broj iznosi 384 što odgovara 24 ms novih zvučnih podataka. Podešavanje varijable
NUMBER\_OF\_NEW\_SLICES\_BEFORE\_INVOKING predstavlja krajnji korak kalibracije sustava.
Potrebno ju je postaviti na najmanji mogući broj koji neće narušavati rad sustava.


