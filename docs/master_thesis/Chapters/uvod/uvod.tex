\chapter{Uvod}
\label{pog:uvod}

%\section{Tiny ML}

Modeli strojnog učenja revolucionarizirali su tehnologiju koju koristimo u svakodnevnom
životu tako što su omogućili računalima učenje iz podataka kojima raspolažu u svrhu
donošenja odluka u situacijama za koje nisu eksplicitno programirana. Tradicionalno
su takvi modeli bili namijenjeni za računala visokih performansi i neograničavajućih
resursa, međutim sve bržim razvojem IoT (engl. \textit{Internet of Things}) područja te
zahvaljujući pristupačnoj cijeni mikrokontrolerskih sustava, počela je prilagodba
modela strojnog učenja za takve sustave. Na prvi pogled dva nespojiva svijeta
su se susrela te pronašla svoju primjenu u raznovrsnim sustavima.
Tiny ML (engl. \textit{Tiny Machine Learning}) je naziv koji se odnosi na implementaciju
modela strojnog učenja na uređaje ograničenih resursa kao što su mobilni telefoni
i mikrokontroleri. Glavne karakteristike modela namijenjenih za takve sustave su
relativno malen memorijski otisak, mogućnost odziva u stvarnom vremenu,
smanjenje potrošnje i internetskog prometa te sigurnost \cite{tinyml}. Sustav
implementiran kroz ovaj rad ima zadatak prepoznati unaprijed zadane glasovne
naredbe te pokrenuti izvršavanje određenog posla vezanog uz specifičnu naredbu.

%\section{Opis problema}

Okruženi smo digitalnim glasovnim asistentima kao što su Googleov Assistant,
Appleova Siri te Amazonova Alexa. Ovakvi sustavi mogu u vrlo kratkom
roku pružiti zatražene informacije i bez ikakvog problema komunicirati s osobom
koja ih koristi. Za prepoznavanje i obradu ljudskog govora i dohvaćanje bitnih
informacija zaduženi su modeli kojima je potrebna velika procesorska moć i dovoljno
prostora za pohranu te se zbog toga taj dio posla odrađuje na serverskim računalima.
Takav sustav podrazumijeva konstantnu internetsku vezu uz stabilan i dugotrajan
izvor električne energije. Kada bi mobilni uređaji slali konstantan tok zvučnih podataka
na server, brzo bi ispraznili bateriju te nepotrebno koristili mobilne podatke za
pristup internetu. Zbog toga su takvi sustavi osmišljeni da čekaju naredbu za
početak komunikacije, a tek onda uspostave vezu sa serverom. Međutim, i dalje
nam ostaje problem konstantne akvizicije ulaznih zvučnih podataka te prepoznavanje
naredbe kao što je ”Hey Google” ili nešto slično. U ovoj situaciji savršenu primjenu
pronašli su procesori izrazito male potrošnje na kojima je moguće implementirati
optimirane modele strojnog učenja. Takav procesor bi konstantno prikupljao podatke
s mikrofona te lokalno, uz pomoć treniranog modela, čekao ključnu riječ nakon
koje bi dao znak cijelom sustavu da se može ”probuditi” iz stanja niske
potrošnje te odraditi svoj posao. Ovakvim pristupom postignuta je efikasnost, niska
potrošnja, brz odaziv, smanjena potrošnja internetskog prometa te možda i
najvažnija stvar - privatnost. Naime, nema potrebe za konstantnim slanjem glasovnih
podataka na server što omogućuje da na server dospiju samo glasovni isječci u
željenim trenucima.
