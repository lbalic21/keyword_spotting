\section{Generiranje značajki}

Generiranje značajki je drugi podsustav spomenut u uvodnom dijelu, također
prikazan na slici \ref{pic:struktura_sustava}. Nakon što je omogućena akvizicija
zvučnih uzoraka, potrebno ih je obraditi kako bi se mogli predati neuronskoj mreži 
na klasifikaciju. Naime, modeli strojnog učenja (kao što je neuronska mreža) rade
s različitim primjerima podataka. Razlikovanje primjera temelji se na odredivanju
različitih značajki ulaznih podataka. Klasičan primjer koji se koristi kako bi se 
približio pojam značajki jest model predikcije cijene nekakve
nekretnine. U tom slučaju značajke koje bi model mogao koristiti su lokacija,
godina izgradnje, površina, razina energetske učinkovitosti i slično. Medutim, što
bi bile značajke našeg ulaznog toka signala? Možemo, ispostavit će se naivno, uzeti
amplitudu svake vrijednosti. U slučaju da promatramo period od jedne sekunde
signala s već spomenutim otipkavanjem od 16 kHz, broj značajki koje bi naš model
morao ”progutati” jest 16000. Obraditi toliku količinu podataka u jako kratkom
vremenu (sustav mora raditi bez konstantno i bez mrtvog vremena) na resursno
ograničenom sustavu kao što je mikrokontroler ne zvuči obećavajuće. Nekako sve 
upućuje na to da je potrebno na neki način prilagoditi ulazni signal, tj. izvući
iz signala bitne informacije i tako smanjiti veličinu podataka. 

\subsection{Model govora}

Kako bismo identificirali kojim značajkama bi bilo korisno opisati ljudski govor,
potrebno je na neki način modelirati nastanak glasa. 



\subsection{MFCC}
\label{MFCCconstruction}





\subsubsection{Window}
\subsubsection{FFT}
\subsubsection{Mel Filterbank}
\subsubsection{DCT}

\subsection{Konstrukcija feature mape}

