\chapter{Arhitektura i implementacija sustava}
\label{pog:struktura_sustava}

Svrha cjelokupnog sustava je detekcija određenog glasovnog uzorka (glasovne naredbe) koja se može koristiti za okidanje obavljanja određenog procesa, pri čemu se prepoznavanje naredbe implementira na ugradbenom računalnom sustavu niske potrošnje i procesorske moći. Također, potrebno je da sustav može pokrenuti
proces u bilo kojem trenutku te bi odaziv trebao biti sa što manjom latencijom, što
znači da cijeli proces prikupljanja zvuka, obrade te izvršavanja akcije mora raditi
kontinuirano u stvarnom vremenu. Centralni dio sustava koji je zadužen za
samo prepoznavanje glasovne naredbe implementiran je u vidu neuronske
mreže trenirane na skupu zvučnih zapisa koji sadrže željene naredbe, a čija je struktura
detaljnije objašnjena u poglavlju koje opisuje samu neuronsku mrežu \ref{pog:neuronska_mreza}
Imajući u vidu spomenute ciljeve, sustav je izgrađen od četiri modularna
podsustava koja su u cijelosti implementirana na odabranom mikrokontroleru, a to su:

\begin{itemize}
    \item prikupljanje zvuka,
    \item izdvajanje značajki,
    \item aktivacija neuronske mreže,
    \item prepoznavanje i aktivacija naredbi.
\end{itemize}

Navedena podjela na podcjeline olakšava prilagodbu cjelokupnog
sustava na različite mikrokontrolerske platforme i porodice, što uključuje drugačiju obradu
prikupljenog zvuka, korištenje neke drugačije neuronske mreže ili samo 
aktiviranje specifične akcije koja će se pokrenuti određenom glasovnom naredbom.
Na slici \ref{pic:struktura_sustava} grafički je prikazan opisani sustav.

\begin{figure}[htb]
    \centering
    \includegraphics[width=1\linewidth]{Chapters/struktura_sustava/struktura_sustava.png} 
    \caption{Struktura sustava na mikrokontroleru\cite{flowchart}}
    \label{pic:struktura_sustava}
\end{figure}

\section{Akvizicija zvuka}
\label{sec:acq}

Podsustav za akviziciju zvuka je usko vezan uz platformu na kojoj
je sustav implementiran zbog toga što je zadužen za komunikaciju 
sa sustavom koji prikuplja stvarne signale sa senzora (mikrofona).
ESP32 Lyrat Development Board \cite{lyrat}, na kojem je sustav
implementiran, na sebi ima već ugrađen mikrofon i audio kodek
s kojim je moguće komunicirati putem I2S protokola. Na slici
\ref{pic:esp} prikazan je blok dijagram ESP32 razvojne platforme.

\begin{figure}[htb]
    \centering
    \includegraphics[width=0.75\linewidth]{Chapters/struktura_sustava/akvizicija/lyrat.png} 
    \caption{ESP32 Lyrat \cite{lyrat}}
    \label{pic:esp}
\end{figure}

Najvažniji dio razvojne platforme je upravo ESP32-WROVER-E mikrokontroler
na kojem je cjelokupni sustav za prepoznavanje govornih naredbi 
implementiran. Upravo on je zadužen za komunikaciju s ES8388 
audio kodekom \cite{es8388}. ES8388 je integrirani sklop koji na spomenutoj
platformi služi za analogno-digitalnu (engl. \textit{ADC}) te digitalno-analognu 
(engl. \textit{DAC}) pretvorbu, tj. upravljanje audio ulazom (mikrofon) te audio
izlazom (AUX konektor). Mikrokontroler putem I2C sučelja konfigurira kodek,
dok konkretne zvučne signale uzima putem I2S sučelja. Budući da je ESP32 Lyrat
platforma prilagođena razvoju audio sustava, konfiguracija i puštanje u pogon
akvizicije zvuka je vrlo jednostavno. U programskom kodu \ref{code:ES8388init}
prikazana je inicijalizacija kodeka.\\

\begin{lstlisting}[language=C++, caption=Inicijalizacija kodeka, label=code:ES8388init]
    audio_board_handle_t board_handle = audio_board_init();
    audio_hal_ctrl_codec(board_handle->audio_hal, AUDIO_HAL_CODEC_MODE_ENCODE, AUDIO_HAL_CTRL_START);
\end{lstlisting}

Sav posao koji se nalazi iza prikazanih naredbi, obavlja ESP-ADF
(engl. \textit{Espressif Audio Development Framework}). To je biblioteka koja pojednostavljuje
razvoj aplikacija vezanih za obradu zvuka kao što je akvizicija, kodiranje i 
dekodiranje različitih formata audio zapisa te komunikacija s računalom ili 
memorijskom karticom.
Nakon inicijalizacije kodeka i definiranja konfiguracije postavki I2S komunikacije,
potrebno je samo pokrenuti protočni akvizicijski sustav (engl. \textit{pipeline}), 
a to je prikazano u programskom kodu \ref{code:pipeline}.\\

\begin{lstlisting}[language=C++, caption=Pokretanje akvizicijskog sustava, label=code:pipeline]
    ESP_LOGI(TAG, "Creating i2s stream");
    audio_element_handle_t i2s_stream_reader;
    i2s_stream_cfg_t i2s_cfg = I2S_STREAM_CFG_DEFAULT();
    i2s_cfg.type = AUDIO_STREAM_READER;
    i2s_cfg.i2s_port = I2S_NUM_0;
    i2s_cfg.i2s_config = i2s;
    i2s_stream_reader = i2s_stream_init(&i2s_cfg);

    ESP_LOGI(TAG, "Creating audio pipeline");
    audio_pipeline_handle_t pipeline;
    audio_pipeline_cfg_t pipeline_cfg = DEFAULT_AUDIO_PIPELINE_CONFIG();
    pipeline = audio_pipeline_init(&pipeline_cfg);
    audio_pipeline_register(pipeline, i2s_stream_reader, "i2s");
    audio_pipeline_run(pipeline);
\end{lstlisting}

Frekvencija otipkavanja postavljena u konfiguraciji iznosi 16000 Hz što
je dovoljno za ovakav tip sustava \cite{wardentinyml}.
Nakon pokretanja akvizicijskog podsustava, sve što je preostalo je uzimati nove
podatke s kraja protočne strukture podsustava. Konstantno uzimanje novih podataka, tj. komunikacija
s ES8388 kodekom, odvojeno je u posebnu dretvu. Na taj način programski je 
odvojena akvizicija sirovih podataka (sirovi u smislu da nisu još obrađeni ni
na kakav način) od ostatka sustava. Ovaj podsustav je proizvođač novih podataka,
dok je ostatak sustava potrošač (također jednodretven). Sve što je preostalo je
ubaciti pouzdan i efikasan način komunikacije između. Za tu svrhu odabran je
kružni međuspremnik (engl. \textit{ring buffer}). On omogućuje asinkrono pisanje u njega
te čitanje iz njega. Implementacija takve strukture dostupna je unutar FreeRTOS-a 
\cite{ringbufferrr}. 


\begin{lstlisting}[language=C++, caption=Razred AudioRecorder, label=code:AudioRecorder]
class AudioRecorder
{
    private:
        uint32_t sampleRate;
        RingbufHandle_t ringBuffer;
        TaskHandle_t captureAudioHandle;
        static void captureAudioTask(void* pvParameters);
    public:
        AudioRecorder(uint32_t sampleRate) : sampleRate(sampleRate), 
                                             ringBuffer(NULL), 
                                             captureAudioHandle(NULL) {}
        uint32_t getSamples(int16_t* samples, size_t numOfSamples);
};
\end{lstlisting}

Opisani podsustav za akviziciju u programskom kodu zapakiran je u razred imena 
AudioRecorder \ref{code:AudioRecorder}. Prema van, razred nudi metodu potpisa
\texttt{uint32\_t getSamples(int16\_t* samples, size\_t numOfSamples)} koja 
omogućuje dohvaćanje proizvoljnog broja \texttt{uint16\_t} podataka jer 
se akvizirani zvučni signal sprema upravo u tom obliku. Zbog ovakve strukture
ovog podsustava, sve što je potrebno u glavnoj petlji sustava je stvaranje objekta
razreda AudioRecorder, alociranje spremnika za dohvaćanje podataka te kontinuirano
pozivanje opisane funkcije koja kao argumente prima pokazivač na alocirani spremnik
te željeni broj audio uzoraka.

Spomenuta modularnost ostvarena je tako što je cijela funkcionalnost akviziranja
novih zvučnih uzoraka sadržana u opisanom razredu. Pokretanje cjelokupnog sustava
na drugoj platformi koja ima drugi audio kodek, drugačiji mikrofon ili se samo 
radi o drugom mikrokontroleru bit će moguće pisanjem novog razreda koji
prati zadano sučelje.
\section{Generiranje značajki}

Generiranje značajki je drugi podsustav spomenut u uvodnom dijelu, također
prikazan na slici \ref{pic:struktura_sustava}. Nakon što je omogućena akvizicija
zvučnih uzoraka, potrebno ih je obraditi kako bi se mogli predati neuronskoj mreži 
na klasifikaciju. Naime, modeli strojnog učenja (kao što je neuronska mreža) rade
s različitim primjerima podataka. Razlikovanje primjera temelji se na odredivanju
različitih značajki ulaznih podataka. Klasičan primjer koji se koristi kako bi se 
približio pojam značajki jest model predikcije cijene nekakve
nekretnine. U tom slučaju značajke koje bi model mogao koristiti su lokacija,
godina izgradnje, površina, razina energetske učinkovitosti i slično. Medutim, što
bi bile značajke našeg ulaznog toka signala? Možemo, ispostavit će se naivno, uzeti
amplitudu svake vrijednosti. U slučaju da promatramo period od jedne sekunde
signala s već spomenutim otipkavanjem od 16 kHz, broj značajki koje bi naš model
morao ”progutati” jest 16000. Obraditi toliku količinu podataka u jako kratkom
vremenu (sustav mora raditi bez konstantno i bez mrtvog vremena) na resursno
ograničenom sustavu kao što je mikrokontroler ne zvuči obećavajuće. Nekako sve 
upućuje na to da je potrebno na neki način prilagoditi ulazni signal, tj. izvući
iz signala bitne informacije i tako smanjiti veličinu podataka. 

\subsection{Model govora}

Kako bismo identificirali kojim značajkama bi bilo korisno opisati ljudski govor,
potrebno je na neki način modelirati nastanak glasa. Prilikom govorne komunikacije,
pluća govornika se pod djelovanjem mišića prsnog koša stišću i potiskuju zrak kroz
vokalni trakt čiji su glavni dijelovi prikazani na slici \ref{pic:glas}.

\begin{figure}[htb]
    \centering
    \includegraphics[width=0.6\linewidth]{Chapters/struktura_sustava/generiranje_znacajki/glas.png} 
    \caption{Presjek glave i osnovni dijelovi vokalnog trakta koji sudjeluju u produkciji govornog signala \cite{petrinovic2002}}
    \label{pic:glas}
\end{figure}

Glasnice (engl. glottis) su vrlo značajan organ u procesu formiranja govora. 
Ponašaju se kao mehanički
oscilator koji prelazi u stanje relaksacijskih oscilacija uslijed struje zraka iz pluća koja kroz njih
prolazi. Na frekvenciju njihovog titranja utječu brojni parametri, a među najznačajnijim su
pritisak zraka iz pluća na ulazu u glasnice i napetost samih glasnica.
Takvim periodičkim titranjem, glasnice formiraju periodičku struju zraka, tj. 
kvazi-periodične impulse (engl. glottal pulse) koja zatim prolaze kroz ostatak 
vokalnog trakta što vodi do stvaranja artikuliranih glasova. U slučaju da 
su glasnice potpuno opuštene, neće
doći do oscilacija i struja zraka iz pluća će neometano prolaziti kroz vokalni trakt 
(tada se ne formira kvazi-periodični impuls, nego je rezultat prolaska zraka kroz glasnice
slučajni šum, a rezultat cijelog procesa je stvaranje neartikuliranih glasova). 

S druge strane, vokalni trakt se ponaša kao filtar koji spektralno mijenja karakteristiku
pobudnog signala (engl. vocal tract frequency response). 
Geometrijom vokalnog trakta, koja se mijenja ovisno o položaju 
artikulatora kao što su jezik, usne, čeljust i resica, bit će određen ton (visina i
spektralni sastav) formiranog signala (govora) \cite{petrinovic2002}. 

Jednostavni model koji se koristi u području obrade prirodnog govora je da se on 
može prikazati kao izlaz iz linearnog, vremenski promjenjivog sustava čija se 
svojstva sporo mijenjaju s vremenom. Međutim, ako se promatraju dovoljno kratki
segmenti govornog signala, svaki se segment može učinkovito modelirati kao izlaz
iz linearnog, vremenski invarijantnog sustava pobuđenog bilo
kvazi-periodičnim impulsima bilo slučajnim šumom (engl. random noise signal).
Opisani sustav može se prikazati jednadžbom \ref{eq:govor}

\begin{equation}
    \label{eq:govor}
    X(f) = E(f) \cdot H(f)
\end{equation}

gdje je:
\begin{itemize}
    \item \(X(f)\) odziv sustava (govor),
    \item \(E(f)\) pobuda (kvazi-periodični impuls),
    \item \(H(f)\) prijenosna funkcija vokalnog trakta.
\end{itemize}

U vremenskoj domeni isti sustav može se prikazati jednadžbom \ref{eq:govor_vremenska}.
Množenju u frekvencijskoj domeni istovjetna je konvolucija u vremenskoj (i obratno!).

\begin{equation}
    \label{eq:govor_vremenska}
    x(t) = e(t) \ast h(t)
\end{equation}

Cilj ovakvog modeliranja je pronaći bitne informacije u govornom signalu. Pretpostavka na
kojoj se temelji daljnji rad je da je skoro sva informacija iz govornog signala sadržana
u prijenosnoj funkciji govornog trakta, tj. pobuda (kvazi-periodični impulsi) ostaje
konstantnna tijekom govora (veliko pojednostavljenje, međutim svi modeli na kojima
se temelji ovo područje zasnivaju na ovoj činjenici \cite{multiplier, emotion, sidhu2024mfcc}).
Zbog toga želimo pronaći način za 
razdvajanje tih dvaju elemenata modela. Ako primijenimo logaritamsku funkciju na
sustav opisan u frekvencijskoj domeni proizlazi \ref{eq:logaritam}.

\begin{equation}
    \label{eq:logaritam}
    \begin{aligned}
        \log(X(f)) &= \log(E(f) \cdot H(f)) \\
        \log(X(f)) &= \log(E(f)) + \log(H(f))
    \end{aligned}
\end{equation}

Prikazanim su uspješno razdvojeni elementi modela, tj. vidljiv je rastav na zbrojnike
ako se primijeni logaritamska skala. Na slici \ref{pic:rastav} prikazan je utjecaj
komponenata modela na konačni rezultat, a to je glas (engl. speech). 

\begin{figure}[htb]
    \centering
    \includegraphics[width=0.6\linewidth]{Chapters/struktura_sustava/generiranje_znacajki/log.png} 
    \caption{Glasovni signal rastavljen na pobudu i odziv vokalnog trakta \cite{sidhu2024mfcc}}
    \label{pic:rastav}
\end{figure}

Jedino što preostaje je pronaći način za što efikasniji opis odziva vokalnog trakta. On će
u konačnici predstavljati zvučne zapise na kojima će model neuronske mreže biti treniran.
U području automatskog prepoznavanja govora te identifikaciji govornika uvelike se koriste
Mel kepstralni vektori.


\subsection{MFCC}
\label{MFCCconstruction}
Mel kepstralni vektori (engl. Mel frequency cepstral coefficients ili MFCC), tj. mjera 
euklidska udaljenosti MFCC vektora jedna je od najčešće korištenih mjera u automatskom 
prepoznavanju govora i govornika \cite{vasilijevic2011perceptual}.





\subsubsection{Window}
\subsubsection{FFT}
\subsubsection{Mel Filterbank}
\subsubsection{DCT}

\subsection{Konstrukcija feature mape}


\section{Aktivacija neuronske mreže}
\section{Prepoznavanje naredbi i aktivacija zadatka}
\label{sec:prepoy}

Ukoliko se mreža aktivira dovoljno često, za očekivati je da će najvjerojatniji izlaz iz 
neuronske mreže u slučaju izgovorene naredbe biti upravo kategorija koja predstavlja tu naredbu. 
Štoviše, naredba bi trebalča biti prepoznata u više uzastopnih iteracija aktiviranja mreže.
Međutim, ponašanje sustava u tom slučaju neće biti onakvo kakvo bi trebalo, a to je da se
određeni posao (zadatak) aktivira isključivo jednom za jednom izgovorenu naredbu. 
Još jedna stvar na koju treba pripaziti je slučajno aktiviranje nekog posla jer zbog
nesavršenosti mreže se može dogoditi da vjerojatnosni izlaz ukazuje na prepoznavanje neke
naredbe iako se u stvarnosti nije izgovorila ista. Međutim, u takvim situacijama se očekuje možda jedna ili dvije takve situacije zaredom,
a ne više njih. Zbog svega navedenog, potrebno je na neki način obraditi tok vjerojatnosnih
izlaza iz neuronske mreže. Kao najbolji način za pokrivanje svih problema pokazalo se 
uprosjećivanje vjerojatnosti pojedinih kategorija. 

Na slici \ref{pic:uml} prikazan je UML dijagram implementiranog podsustava. Upravljački dio posla
obavlja se unutar razreda \texttt{CommandRecognizer} koji je zadužen za aktiviranje obavljanja 
konkretnog posla koji je povezan s govornom naredbom. 
Sadrži listu pokazivača na objekte čiji razredi implementiraju 
sučelje \texttt{Command}. \texttt{Command} predstavlja sučelje (apstraktni razred) što znači da
nije moguće konstruirati objekt tog razreda, nego da je potrebno naslijediti razredom koji 
implementira apstraktnu metodu \texttt{virtual void execute(float probability)}. Ta 
metoda je zadužena za obavljanje konkretnog zadatka. Ovakav strukturni
obrazac daje mogućnost vrlo lakog implementiranja novih vrsta zadataka (sve što je potrebno je 
napraviti novi razred koji implementira sučelje \texttt{Command}, tj. nadjačava
spomenutu metodu). Trenutno su implementirane
dvije vrste naredbi (zadataka): \texttt{BlankCommand} koja ne radi ništa (koristi se za kategorije
"pozadina" i "nepoznato") i \texttt{PrintCommand} koja ispisuje ime naredbe i prosječnu 
vjerojatnost pojavljivanja naredbe.

\begin{figure}[htb]
    \centering
    \includegraphics[width=0.8\linewidth]{Chapters/struktura_sustava/prepoznavanje_naredbi/commands.png} 
    \caption{Podsustav za prepoznavanje naredbi i aktivaciju zadataka\cite{flowchart}}
    \label{pic:uml}
\end{figure}

Prilikom konstrukcije objekta razreda koji implemntira sučelje \texttt{Command} potrebno je postaviti
parametre za prepoznavanje naredbe (oni se utvrđuju eksperimentalno). Parametrom \texttt{historySize}
bira se broj uzastopnih vrijednosti vjerojatnosti pojave naredbe nad kojima se računana prosjek,
a parametar \texttt{threshold} postavlja prag koji prosječna vrijednost mora prijeći kako bi se naredba
aktivirala. Ovim se postiže odvojeno kalibriranje parametara za svaku naredbu (potrebno
je zbog podataka nad kojima se mreža uči te se zbog toga može dogoditi da se određene
naredbe prepoznaju lakše ili teže od drugih).
Uz to, objekt razreda \texttt{CommandRecognizer} brine o tome da se naredba aktivira
samo ako je prošlo određeno vrijeme nakon zadnje aktivacije. To se postiže parametrom 
\texttt{COOL\_DOWN\_PERIOD\_MS} koji je definiran u datoteci Configuration.hpp \ref{add:config}.
Jednom konstruirani objekt naredbe potrebo je dodati 
u objekt klase CommandRecognizer metodom \texttt{bool addCommand(Command* command)} 
kako bi on bio svjestan postojanja te naredbe (broj izlaza iz neuronske mreže mora odgovarati 
broju dodatnih naredbi ovim putenm). Svaka konkretna implementacija metode 
\texttt{virtual void execute(float probability)} ne smije trajati predugo jer će unijeti 
kašnjenje u cjelokupni sustav. Ideja je da takva naredba samo pokrene duži posao za koji će onda
biti zadužen neki drugi sustav.

U ovom trenutku opisana je cjelokupna petlja mikrokontrolerskog sustava
koja se neprestano iznova izvršava (slika \ref{pic:struktura_sustava}).
Jedina stvar koja je dosad uzimana kao "crna kutija" je sama neuronska mreža koja
je u biti okosnica cijelog sustava. Sljedeće poglavlje posvećeno je upravo njoj.


