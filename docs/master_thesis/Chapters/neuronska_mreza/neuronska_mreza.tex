\chapter{Neuronska mreža za prepoznavanje govornih naredbi}
\label{pog:neuronska_mreza}

\section{Načela rada neuronskih mreža}

Neuronska mreža ili, preciznije, umjetna neuronska mreža (engl. \textit{neural network}) je 
računalni model inspiriran
biološkom strukturom neurona u mozgu. Predstavlja jedan od najkorištenijih modela 
u dubokom učenju. Sastoji se od čvorova (neurona) i jednosmjernih
veza između njih (sinapsa) koji tako tvore usmjereni graf. Čvorovi su grupirani u slojeve,
a svaki od njih je povezan s čvorovima iz susjednog sloja na određeni način. Način na koji su
određeni slojevi međusobno povezani određuje vrstu sloja. 

Jednostavan primjer strukture neuronske mreže je mreža izgrađena od potpuno povezanih slojeva 
(engl. \textit{fully connected layer ili dense layer}). Oni se često koristi kao osnovni građevni blok u 
umjetnim neuronskim mrežama \cite{dense}.
U potpuno povezanom sloju svaki čvor jednog sloja povezan je sa svakim čvorom susjednog sloja.
Ovakva struktura omogućava mreži fleksibilno učenje složenih odnosa između ulaznih i izlaznih podataka.

\begin{figure}[htb]
  \centering
  \includegraphics[width=0.5\linewidth]{Chapters/neuronska_mreza/dense_layer.png} 
  \caption{Potpuno povezani slojevi \cite{dense}}
  \label{pic:dense_layer}
\end{figure}

Na slici \ref{pic:dense_layer} prikazana je struktura neuronske mreže koja se sastoji od ulaznog sloja,
dva potpuno povezana sloja te izlaznog sloja. Srednji slojevi (svi osim ulaznog i izlaznog)
se još nazivaju i skriveni slojevi jer kada se koristi model neuronske mreže, obično se
na njega gleda kao na crnu kutiju koja na ulazu prima vrijednosti te na izlazu daje vrijednosti
izračunate kroz sve skrivene slojeve \cite{fully_connected}.

Svaka veza između pojedinih čvorova ima određenu vrijednost koju nazivamo težina, a svaki čvor
zapravo predstavlja funkciju koja može aktivirati svoj izlaz i vezu sa sljedećim čvorom.

\begin{equation}
  \label{eq:neuron_activation}
  a = f\left(\sum_{i=1}^n w_i x_i + b\right)
\end{equation}

Jednadžba \eqref{eq:neuron_activation} modelira ponašanje pojedinog čvora u mreži. Aktivacijska
funkcija \( f \) je vrlo bitna u odvajanju bitnih od nebitnih utjecaja pojedinih čvorova na sljedeći
čvor. Također, ona omogućava modeliranje složenijih nelinearnih odnosa \cite{activation_fcn}.
Kada bi čvor bio modeliran
bez aktivacijske funkcije, svako preslikavanje koje bi činio bi bilo linearno, a zbog toga što
svaka kompozicija linearnih funkcija daje opet linearnu funkciju, cjelokupna mreža ne bi bila sposobna
modelirati kompleksnije odnose. Sastavnice modela čvora su sljedeće:

\begin{itemize}
  \item \( a \): izlazna vrijednost čvora (aktivacija)
  \item \( f \): aktivacijska funkcija
  \item \( w_i \): težina i-tog ulaznog čvora (čvor u prijašnjem sloju)
  \item \( x_i \): vrijednost i-tog ulaznog čvora (njegova aktivacija)
  \item \( b \): pomak
  \item \( n \): broj čvorova u prijašnjem sloju koji imaju vezu s modeliranim čvorom
\end{itemize}

Povezivanjem više ovako definiranih čvorova gradi se neuronska mreža. Ulaz u neuronsku mrežu 
je informacija na temelju koje će na izlazu iz mreže biti vrijednost izračunata s pomoću svih
slojeva u mreži. Kako bi vrijednosti na izlazu iz mreže imale smisla, tj. davale korisnu
informaciju, potrebno je trenirati mrežu. Treniranje mreže, u općem slučaju nadziranog strojnog
učenja, podrazumijeva korištenje označenog skupa podataka koji je istog oblika kao i podaci
koji će biti na ulazu u mrežu tijekom korištenja same mreže. Arhitektura mreže (vrsta, veličina
i broj slojeva) određena je prije samog treniranja, dok se težine, pomaci te samim time i 
razine aktivacija uče, tj. treniraju. Treniranje je proces u kojem neuronska mreža na svoj ulaz dobiva
označeni skup podataka (označeni skup predstavlja podatke za koje znamo što bi mreža trebala
dati na izlazu) te provjerava koliko izlazi odstupaju od prave oznake. Na početku su sve težine
uglavnom inicijalizirane na nulu. Podatak se predaje ulaznom sloju, prolazi kroz sve skrivene
slojeve te na izlazu mreža izbacuje određenu vrijednost koja se s očekivanom uspoređuje s pomoću
funkcije gubitka. Takva funkcija predstavlja koliko izlazi iz mreže odstupaju od očekivanih.

%Na slici \ref{pic:feedforward} prikazana je jednostavna neuronska mreža čiji izlaz daje određenu
%vrijednost funkciji gubitka.
%\begin{figure}[htb]
%  \centering
%  \includegraphics[width=0.5\linewidth]{Chapters/neuronska_mreza/feed_forward.png} 
%  \caption{Dense}
%  \label{pic:feedforward}
%\end{figure}


Cilj svakog treniranja jest smanjiti vrijednost funkcije gubitka. Stoga se sve težine u mreži
ažuriraju tako da njihove promjene pomaknu trenutačno stanje mreže u smjeru negativne derivacije
funkcije gubitka (gradijentni spust). Takvim pristupom, mreža kroz iteracije s novim podacima smanjuje funkciju gubitka
(efektivno daje sve točnije predikcije). Težine se ažuriraju od izlaznog sloja prema ulaznom 
(engl. \textit{backpropagation}) jer na izlaz pojedinog sloja utječu njegovi ulazi, tj. izlazi prijašnjeg
sloja. Naime, izlaz svakog sloja je funkcija izlaza prijašnjeg sloja, a kad se izračuna derivacija
funkcije, promjenom njenih ulaza poznato je u kojem smjeru će se mijenjati vrijednost same funkcije.


%Na slici \ref{pic:backprop} prikazana je mreža i način ažuriranja težina "unazad".
%\begin{figure}[htb]
%  \centering
%  \includegraphics[width=0.5\linewidth]{Chapters/neuronska_mreza/backprop.png} 
%  \caption{Dense}
%  \label{pic:backprop}
%\end{figure}

Bez smanjenja općenitosti, funkcija gubitka prikazana je na trodimenzionalnom grafu
na slici \ref{pic:descent}. Složene arhitekture mreža će imati više dimenzija zbog većeg broja
težina \( w_i \). Cilj svakog treniranja mreže je doći što bliže minimumu ovakve funkcije.
Svakom iteracijom mreža se pomiče sve bliže minimumu, a takva vrsta optimizacije naziva se
gradijentni spust (engl. \textit{gradient descent}).

\begin{figure}[htb]
  \centering
  \includegraphics[width=0.5\linewidth]{Chapters/neuronska_mreza/descent.png} 
  \caption{Funkcija gubitka \cite{desc}}
  \label{pic:descent}
\end{figure}


%ostala poglavlja
\section{CNN}
\label{sec:cnn}

Konvolucijska neuronska mreža (eng. Convolutional neural network, CNN) je vrsta 
umjetne neuronske mreže koja je pogodna je za obradu podataka s rešetkastom
topologijom, a najviše se koristi za rješavanje problema iz područja klasifikacije slika
te računalnog vida. Inspirirane su načinom funkcioniranja moždanog korteksa 
zaduženog za vid kod sisavaca \cite{pycodemates}. Obrada slike koju vide sisavci
funkionira hijerarhijski, tj. u mozgu se ne obrađuje cjelokupna slika odjednom,
nego postoje jednostanije stanice koje su zadužene za prepoznavanje osnovnijih
oblika koji se nakon toga stapaju u složenije i složenije. Na poslijetku 
organizam je u mogućnosti prepoznati cjelokupnu sliku koju gleda očima.
Međutim, kakve veze ima klasifikacija slika s problemom prepoznavanja glasovnih naredbi?
Naime, generirana matrica značajki iz poglavlja \ref{sec:gen} predstavlja upravo dvodimenzionalnu
sliku koja se može koristiti kao ulaz u konvolucijsku neuronsku mrežu. 

Računalni modeli koji koriste strukturu sličnu opisanoj mogu iz
podataka koji su u takvom obliku izvući značajke samostalno što znači da nema
potrebe za korištenjem metoda koje eksplicitno izvlače bitne značajke iz podataka
\cite{1}. Arhitektura najjednostavnije konvolucijske neuronske mreže
uključuje:

\begin{itemize}
    \item \(Ulaz\) (eng. Input Layer): ulazni sloj modela, prima matrični podatak
    \item \(Konvolucijski\) \(sloj\): osnovni sloj modela. Njegov glavni zadatak
          je ekstrakcija značajki iz ulaznih podataka. Vidi \ref{sub:conv}.
    \item \(Sloj\) \(za\) \(poduzorkovanje\): smanjuje dimenzionalnost (vidi \ref{sub:pooling})
    \item \(Potpuno\) \(povezani\) \(sloj\): povezuje značajke s klasifikacijom (vidi \ref{sub:dense})
    \item \(Izlaz\) : izlazni sloj, daje vjerojatnosti klasifikacije (vidi \ref{sub:output})
\end{itemize}

\begin{figure}[htb]
    \centering
    \includegraphics[width=0.5\linewidth]{Chapters/neuronska_mreza/CNN/cnn.png} 
    \caption{Jednostavna CNN \cite{1}}
    \label{pic:cnn}
\end{figure}

Na slici \ref{pic:cnn} prikazana je opisana struktura. Prva tri sloja grupirana su
u dio koji služi za ekstrakciju značajki iz ulaznih podataka, a posljednja dva
sloja služe za klasifikaciju. Složenije arhitekture mreže mogu imati veći broj
konvolucijskih slojeva (nakon svakog se nalazi sloj za poduzorkovanje) te veći broj
složenijih ili manje složenih potpuno povezanih slojeva.

\subsection{Konvolucijski sloj}
\label{sub:conv}

Najbitniji dio konvolucijske neuronske mreže je konvolucijski sloj zbog toga što se
u njemu događa konvolucija. Konvolucija (u neuronskim mrežama) je proces kojim 
iz ulazne matrice podataka (slike) na izlazu dobijemo matricu značajki ili
mapu značajki. Neka ulazna matrica bude oblika \( x \in M_{mn}(\mathbb{R}) \).
Umjesto težina (kao kod potpuno povezanog sloja), konvolucijski sloj koristi
matricu \( \omega \in M_{pr}(\mathbb{R}) \) koju nazivamo filtar ili
jezgra (eng. kernel) \cite{keras_layers}. Svaki konvolucijski 
sloj može imati proizvoljan broj filtara. Izlazna (u ovom slučaju dvodimenzionalna)
mapa značajki tada se računa na sljedeći način:

\begin{equation}
h = \omega * x,
\end{equation}

pri čemu \( * \) označava operaciju konvolucije te vrijedi:

\begin{equation}
h(i, j) = (\omega * x)(i, j) = 
\sum_{k=0}^{p-1} \sum_{l=0}^{r-1} x(i + k, j + l) \omega(k, l).
\end{equation}

Formula koja se zapravo koristi naziva se unakrsna korelacija, međutim, zbog sličnosti s 
formulom konvolucije, mreža nosi takav naziv \cite{gracan2020}. Na slici 
\ref{pic:convolution} prikazan je proces koji se događa tijekom prolaska 
ulaznog podatka kroz konvolucijski sloj. Ulaz (eng. input) se konvolucijski množi
s jezgrom kako bismo dobili izlaznu matricu značajki \cite{cnn_how}. U slučaju prikazanom na slici,
ulazna slika je veličine \(4 \times 4\), dok je jezgra veličine \(3 \times 3\). Prvo se podmatrica ulaznog
podatka veličine jednake veličini jezgre (\(3 \times 3\)) skalarno množi s jezgrom. Izlaz je 
skalarni umnožak na koji se može dodati konstantna vrijednost (eng. bias). Nakon
toga se jezgra pomiče po ulaznoj matrici, tj. sljedeći element izlazne matrice je
skalarni umnožak jezgre i sljedeće podmatrice ulaznog podatka. Koliko će se jezgra
pomaknuti određuje pomak (eng. stride). U slučaju na slici \ref{pic:convolution}
pomak iznosi jedan.

\begin{figure}[htb]
      \centering
      \includegraphics[width=0.5\linewidth]{Chapters/neuronska_mreza/CNN/convolution.png} 
      \caption{Konvolucija \cite{convolution}}
      \label{pic:convolution}
\end{figure}

Rezultat opisanog procesa je mapa značajki koja je manja od ulazne, a njema veličina
obrnuto proporcionalno ovisi o veličini jezgre te pomaku \cite{cnn_whatis}. 

Slično procesu koji se odvija u mozgu čovjeka, opisani struktura omogućava hijerarhijsko učenje.
Naime, svaki konvolucijski sloj sadrži jezgre koje su zadužene za lokalno pretraživanje
određenih uzoraka (upravo konvolucijom izvlačimo stvari koje su slične između različitih
ulaznih slika). Koje jezgre se trebaju koristiti? Vrlo jednostavno, proces učenja (treniranja
mreže) će prepoznati lokalne sličnosti između različitih primjera! Ako strukturiramo mrežu
tako da se sastoji od više uzastopnih konvolucijskih slojeva, mreža će prvo naučiti najjednostavnije
oblike, a zatim u sljedećem sloju takvim oblicima slagati složenije uzorke. Također,
još jedna prednost konvolucijskog sloja je u dijeljenju parametara. Takav sloj nema vezu 
svakog neurona sa svakim ulaznim, nego se težine dijele unutar određene jezgre. Zapravo se
cijeli proces učenja svodi na nalaženje odgovarajućih jezgri koje će prepoznati uzorke \cite{pycodemates}.
Evo uzmimo, na primjer, dvodimenzionalnu ulaznu sliku. Broj pametara takvog dvodimenzionalnog
sloja iznosit će:

\begin{equation}
    N = (n \cdot m \cdot C_{\text{in}} + 1) \cdot C_{\text{out}}
\end{equation}

Gdje:
\begin{itemize}
    \item \(N\): ukupni broj parametara
    \item \(n\) i \(m\): visina i širina filtra (jezgre)
    \item \(C_{\text{in}}\): Broj ulaznih kanala (npr. 1 za crno-bijele slike, 3 za RGB slike).
    \item \(+1\): konstanta svakog filtra
    \item \(C_{\text{out}}\): Broj filtara (odnosno mapa izlaznih značajki).
\end{itemize}

Kako bismo bolje dočarali razliku u broju parametara između ovakog sloja i potpuno povezanog sloja,
uzmimo za primjer sliku \ref{pic:convolution}. Ulazna slika je veličine \(4 \times 4\),
jezgra \(3 \times 3\), a izlaz 
\(2 \times 2\). Neka je slika jednokanalna, a broj jezgri jedan (sve kao na slici). Konvolucijski sloj
će imati 10 parametara, dok će potpuno povezani sloj (16 ulaznih vrijednosti, 4 izlazne te
4 konstante za svaki neuron) imati 68 parametara! Formula za broj parametara u tom slučaju
je sljedeća:

\begin{equation}
    N = n_{\text{in}} \cdot n_{\text{out}} + n_{\text{out}}
\end{equation}

Gdje:
\begin{itemize}
    \item \(N\): ukupni broj parametara
    \item \(n_{\text{in}}\): broj ulaznih neurona
    \item \(n_{\text{out}}\): broj izlaznih neurona
\end{itemize}

Također, kao što je opisano u poglavlju \ref{pog:neuronska_mreza} vrijednost neurona (čvora) na
izlazu it sloja potrebno je provući kroz aktivacijsku funkciju. Postoje različite vrste funkcija
koje se koriste u različitim granama strojnog i dubokog učenja \cite{activation_fcn}, a za
primjenu u konvolucijskim slojevima, najefektivnija se pokazala ReLu (eng. Rectified linear units)
\cite{relu}. To je funkcija koja vraća nulu ako joj je ulaz negativan, a za svaki pozitivan
ulaz samo prosljeđuje istu vrijednost na izlaz. Modelirana je formulama \ref{eq:relu1} i 
\ref{eq:relu2}, a prikazana je na slici \ref{pic:relu}. 

\begin{equation}
    f(x) = \max(0, x)
    \label{eq:relu1}
\end{equation}

\begin{equation}
    f(x) = 
    \begin{cases} 
        0, & \text{ako } x < 0, \\
        x, & \text{ako } x \geq 0.
    \end{cases}
    \label{eq:relu2}
\end{equation}

\begin{figure}[htb]
    \centering
    \includegraphics[width=0.5\linewidth]{Chapters/neuronska_mreza/CNN/relu.png} 
    \caption{ReLu \cite{relu}}
    \label{pic:relu}
\end{figure}

Prednosti ReLu funcije nad nad drugim aktivacijskim funkcijama je u tome za negativne
ulaze uopće ne aktivira neuron što izuzetno povećava efikasnost. Druga prednost je u tome
što izlaz nelinearno raste s porastom ulazne vrijednosti što znači da nikad neće ući
u zasićenje. Ta osobina je važna jer utječe na izgled funkcije gubitka te ubrzava
konvergenciju gradijentnog spusta prema minimumu funkcije gubitka \cite{activation_fcn}.


\subsection{Sloj za poduzorkovanje}
\label{sub:pooling}
Sloj za poduzorkvanje (eng. Pooling layer): služi smanjenju dimenzionalnosti matrice
značajki na izlazu iz konvolucijskog sloja. Najčešće korištene tehnike su maksimalno 
(eng. max pooling) i prosječno poduzorkovanje (eng. average pooling). Radi tako da više 
susjednih vrijednosti spoji u jednu te tako na svom izlazu da matricu manjih dimenzija 
\cite{pooling1}. Na taj način postupno smanjuje broj parametara, smanjuje broj
operacija potrebnih za daljnje računanje te ono najbitnije, kontrolira
prenaučenost \cite{cnn_whatis}. Na slici \ref{pic:pooling} prikazane su obje navedene vrste poduzorkovanja.

\begin{figure}[htb]
    \centering
    \includegraphics[width=0.5\linewidth]{Chapters/neuronska_mreza/CNN/pooling.png} 
    \caption{Poduzorkovanje \cite{pooling1}}
    \label{pic:pooling}
\end{figure}

Prosječno poduzorkovanje izglađuje sliku (eng. smoothing). Zbog toga oštri detalji slike
mogu biti izgubljeni što znači da se određene značajke možda neće prepoznati kada se 
koristi ova metoda. Maksimalno uzorkovanje odabire najsvjetlije piksele iz slike,
a oni su se pokazali kao najbitnije značajke jer daju najbolje rezultate \cite{cnn_whatis}.
Suprotno tome, minimalno uzorkovanje (eng. min pooling) odabralo bi najtamnije piksele,
međutim ono se najrjeđe koristi.

\subsection{Sloj za poravnavanje}
Sloj za poravnavanje (eng. flatten layer) je sloj koji dolazi nakon posjednjeg sloja
za poduzorkovanje. Njegova jedina zadaća je poravnati izlaz iz prijašnjeg sloja.
Ovaj sloj ništa ne računa, ništa ne uči, jedina zadaća mu je od ulaznih mapa značajki
napraviti jedan vektor koji je onda moguće povezati na potpuno povezani sloj. Zbog
toga se često izostavi iz skica koje prikazuju strukture CNN-ova kao što je slučaj
na slici \ref{pic:cnn}. Na slici \ref{pic:flatten} prikazana je uloga ovog sloja.

\begin{figure}[htb]
    \centering
    \includegraphics[width=0.5\linewidth]{Chapters/neuronska_mreza/CNN/flatten.png} 
    \caption{Sloj za poravnjavanje \cite{flatten}}
    \label{pic:flatten}
\end{figure}


\subsection{Potpuno povezani sloj}
\label{sub:dense}
Potpuno povezani sloj (eng. Fully Connected Layer) detaljnije je pojašnjen u poglavlju
\ref{pog:neuronska_mreza}. Nakon prijašnjih slojeva koji su služili za izvlačenje
značajki iz ulaznih podataka, na red dolazi klasifikacija. Budući da je prethodnik
prvom ovakvom sloju sloj za poravnavanje, nemamo problem sa spajanje ovog sloja na dosad
objašnjenu strukturu. Uloga ovog sloja (ili više ovakvih slojeva) je, najjednostavnije
rečeno, klasifikacija. Značajke naučene tijekom konvolucije se ovdje predaju gustoj mreži
neurona koja je sposobna odraditi posao do kraja, tj. naučiti kako različite značajke
pridonose određenoj izlaznoj klasi.

\subsection{Izlazni sloj}
\label{sub:output}
Izlazni (eng. Output Layer) je posljednji sloj potpuno povezanog sloja (a ujedno i cijele
neuronske mreže). Ima onoliko neurona koliko želimo imati klasa, pojedina vrijednost
neurona predstavlja vjerojatnost pripadnosti klasi. Da bi to stvarno funkcioniralo na takav
način, potrebno je odrediti prikladnu aktivacijsku funkciju. Funkcija koja radi baš to 
naziva se funkcija softmax. 
Formalno, \( \text{softmax} : \mathbb{R}^n \to \mathbb{R}^n \),
gdje je \( k \)-ta komponenta izlaznog vektora definirana kao:
\begin{equation}
\text{softmax}_k(x_1, \dots, x_n) = \frac{\exp(x_k)}{\sum_{j} \exp(x_j)}
\end{equation}

Funkcija softmax radi dvije ključne stvari:
\begin{itemize}
    \item Normalizira sve vrijednosti tako da njihov zbroj bude \( 1 \), tj. izlazni vektor 
        predstavlja distribuciju vjerojatnosti.
    \item Pojačava veće vrijednosti (čini ih dominantnijima) i smanjuje manje vrijednosti.
\end{itemize}

Funkcija nosi naziv \textit{softmax} jer odgovara funkciji \( \max \), ali je "meka" u smislu
da je neprekidna i diferencijabilna, za razliku od klasične \( \max \) funkcije 
\cite{snajder2023logreg}.

\begin{figure}[htb]
    \centering
    \includegraphics[width=0.6\linewidth]{Chapters/neuronska_mreza/CNN/softmax.png} 
    \caption{Softmax \cite{snajder2023logreg}}
    \label{pic:softmax}
\end{figure}


\subsection{Poznate arhitekture konvolucijskih mreža}
Arhitektura konvolucijske mreže je ključni faktor koji određuje njene performanse i
učinkovitost. Broj konvolucijskih slojeva, izgled istih (broj filtara, njihova veličina,
pomak), vrsta slojeva za poduzorkovanje te broj i veličina potpuno povezanih slojeva znatno
utječu na brzinu izvođenja i preciznost klasifikacije. Naravno, ne postoji jedan recept koji
najbolje radi na svim vrstama ulaznih podataka, nego različite arhitekture daju bolje rezultate
u određenim situacijama. Određene arhitekture su kroz povijest ostale zapamćene zbog
toga kako su utjecale na duboko učenje\cite{indian}:

\begin{itemize}
    \item \textbf{LeNet-5 (1998):} 
    CNN sa 7 slojeva dizajnirana za klasifikaciju rukom pisanih brojeva na slikama 
    veličine \(32 \times 32\) piksela u sivim tonovima. Koristila se u bankama za čitanje čekova 
    i bila je prvi značajan korak u korištenju CNN-a u stvarnom svijetu.

    \item \textbf{AlexNet (2012):} 
    Proširena verzija LeNet-a s dubljom arhitekturom (5 konvolucijskih i 3 potpuno povezana 
    sloja). Prva mreža koja je koristila ReLU aktivaciju za brže treniranje. Značajno smanjila
    stopu pogreške na ILSVRC natjecanju i popularizirala duboko učenje.

    \item \textbf{GoogleNet (Inception V1) (2014):} 
    22-slojna mreža s inovativnim \emph{inception module}, koji koristi male konvolucije za 
    smanjenje broja parametara (sa 60 milijuna na samo 4 milijuna). Pobjednik ILSVRC 2014 s 
    top-5 pogreškom manjom od 7\%. Performanse su usporedive s ljudskim prepoznavanjem slika.

    \item \textbf{VGGNet (2014):} 
    Mreža sa 16 konvolucijskih slojeva koja koristi samo \(3 \times 3\) konvolucije s povećanim
    brojem filtara. Iako je jednostavna u dizajnu, ima 138 milijuna parametara, što je čini 
    računalno zahtjevnom za treniranje i implementaciju.
\end{itemize}
\section{Skup podataka za treniranje}
\label{sec:dataset}

Prvi korak u izgradnji sustava koji koristi bilo kakav tip modela strojnog učenja
je odabir i priprema prikladnog skupa podataka na kojem će se taj model trenirati.
Budući da je cilj sustava prepoznavanje govornih naredbi (eng. keyword spotting),
savršen otvoreni skup podataka je skup za treniranje prepoznavanja ograničenog
skupa naredbi \cite{speechcommandsv2}. Riječ je o skupu koji
se sastoji od oko 105000 zvučnih isječaka duljine oko jedne sekunde (frekvencija
zapisa je 16 kHz) u kojima ljudi
izgovaraju jednu od 35 različitih riječi. Također, skup ima nekoliko vrsta 
pozadinske buke koja je ključna za rad ovakvog sustava u pravom svijetu. 
U prikupljanju podataka sudjelovalo je oko 2600 ljudi iz cijelog svijeta.
U privitku je prikazana tablica koja prikazuje od kojih se riječi skup sastoji
te koliko snimaka pojedinih riječi postoji \ref{tab:word_frequency}.


\section{Priprema podataka}
\label{sec:data}

Nakon odabira skupa na kojem će model biti treniran, potrebno je pripremiti
podatke. Zbog ograničenih resursa na mikrokontrolerskom sustavu, nije moguće
(a niti potrebno za konkretnu namjenu) izgraditi sustav koji će moći
prepoznati sve riječi iz skupa. Potrebno je odabrati samo podskup
za koji će sustav uspješno klasificirati izgovorenu riječ. Cjelokupni
proces pripreme podataka, treniranja i validacije modela popraćen je
Jupyter bilježnicom koja se nalazi u GitHub repozitoriju
\cite{balic_keyword_spotting}.

Projektirani sustav mora moći prepoznati naredbe koje su izgovorene.
Zbog toga, mora biti sposoban odbaciti sve ono što nije naredba koja se
nalazi u odabranom skupu. Iz toga proizlazi da jedna od kategorija (klasa)
na kojoj će model biti treniran jest pozadinska buka. Odabrani skup podataka
sadrži zvučne zapise kao što je zvuk perilice posuđa, zvuk vode koja teče
iz slavine, bijeli šum te ružičasti šum. Bijeli šum (eng. white noise) je vrsta
signala koji u sebi sadrži sve frekvencije i sve imaju isti intenzitet,
dok se ružičasti šum također sastoji od svih frekvencija, ali veći intenzitet
imaju niže frekvencijske komponente \cite{noise}. Međutim, zbog toga što skup
podataka zadrži
svega nekoliko minuta takvih zvučnih zapisa, potrebno je na neki način dopuniti
taj podskup podataka. Naime mreža koju treniramo će preferirati neku od klasa
ako takvih primjera ima mnogo više od primjera ostalih klasa. Drugim riječima,
skup podataka za treniranje mora biti balansiran (primjera iz svake klase treba
biti otprilike jednak broj) \cite{balance}. Uz to, za rad u stvarnom svijetu, nije loše u skup
dodati zvučni zapis snimljen upravo na sustavu koji će i akvizirati podatke iz okoline.
Uz snimke pozadinskih zvukova iz skupa podataka, dodane su snimke snimljenje upravo
na sustavu gdje će sve biti implementirano, a zadrže karakteristične pozadinske
zvukove okruženja u kojem će se nalaziti uređaj. Budući da su svi zvučni zapisi 
riječi u skupu podataka duljine od otprilike jedne sekunde, pozadinske zvukove
je potrebno izrezati na identičnu duljinu kako bi mreža mogla primati vrlo 
precizno definiranu vrstu ulaznih podataka. O broju riječi iz klasa koje su odabrane 
kao naredbe ovisit će koliko trebamo imati isječaka koji će predstavljati
klasu pozadinskih zvukova. Metoda kojom lako možemo "umnožiti" broj pozadinskih 
zvukova zove se augmentacija.

Augmentacija zvuka je proces u kojem se razičitim metodama može izmijeniti 
zvučni zapis. U ovom slučaju koristi se za povećanje broja snimaka na kojima
je pozadisnka buka. Umjesto se samo kopiraju uzorci, ovakvim promjenama stvaraju
se novi audio zapisi slični onima od kojih su nastali, međutim dovoljno različiti 
da povećaju robusnost sustava. Metode korištene u umnožavanju danih zvukova
pozadinske buke su nasumično ubrzavanje i povećanje ili smanjenje glasnoće,
dodavanje jeke (preklapanje originalnog zapisa s istim ali pomaknutim u vremenu)
te okretanjem uzoraka u snimci (nova snimka je obrnuta od originala). Funkcija
za augmentaciju prikazana je u odsječku programskog koda \ref{code:augmentation}.

\begin{lstlisting}[language=Python, caption=Augmentacija zvuka, label=code:augmentation]
def augment_audio(audio: AudioSegment) -> AudioSegment:
    #Start with the original audio
    augmented = audio
    #Randomly apply speed adjustments (time stretching)
    if random.random() > 0.5:
        augmented = speedup(augmented, playback_speed=random.uniform(1.1, 1.5))
    #Normalize volume to ensure consistent loudness
    augmented = normalize(augmented)
    #Add random volume adjustments (increase or decrease volume)
    if random.random() > 0.5:
        volume_change = random.uniform(-5, 5)  #Random volume change in dB
        augmented = augmented + volume_change
    #Add echo (simulated by overlapping the original with a delayed copy)
    if random.random() > 0.5:
        delay_ms = random.randint(100, 500)  #Random delay in milliseconds
        echo = augmented - random.uniform(5, 10)  #Lower volume for the echo
        augmented = augmented.overlay(echo, position=delay_ms)
    #Optionally, you could reverse the audio for more variation
    if random.random() > 0.5:
        augmented = augmented.reverse()
return augmented
\end{lstlisting}


Druga kategorija mora biti sastavljena od kombinacije različitih riječi za koje
ne želimo klasifikaciju, tj. nisu odabrane u podskup naredbi. Ime te kategorije
će biti "nepoznato" (engl. unknown), a predstavljat će sve riječi koje nisu
naredbe cjelokupnog sustava. Ostale kategorije bit će riječi odabrane kao naredbe
sustava što  znači da će ukupan broj klasifikacijskih kategorija biti za dva veći 
od broja odabranih naredbi (broj naredbi + pozadinska buka + nepoznato).

Odabir naredbi koje sustav može prepoznati je proizvoljan, a za primjer na kojem
će daljnja obrada biti opisana odabrane su naredbe "yes" i "no". Zbog toga ukupni broj
klasifikacijskih kategorija iznosi četiri. Uz ostale konfiguracijske parametre,
odabir naredbi omogućen je na početku Jupyter bilježnice za treniranje neuronske mreže.
Nakon toga formira se mapa s četiri datoteke koje prestavljaju 4 klasifikacijske kategorije.
Broj zapisa u svakoj od kategorija odgovarat će broju zapisa u najmalobrojnijoj 
kategoriji upravo zbog spomenute potrebe za balansiranim skupom podataka za treniranje
(višak naredbi u nekoj od kategorija koje predstavljaju naredbe se neće koristiti, broj
zapisa u kategoriji pozadinske buke generirat će se augmentacijom po potrebi, a broj
nasumično odabranih zapisa u kategoriji "nepoznato" moguće je napraviti proizvoljno
velikim).

Učitavanje zvučnih zapisa iz datotečnog sustava pojednostavljeno je korištenjem
TensorFlow biblioteke, a prikazano je u odsječku programskog koda \ref{code:load}.

\begin{lstlisting}[language=Python, caption=Učitavanje zvučnih zapisa, label=code:load]
train_dataset, validation_dataset = tf.keras.utils.audio_dataset_from_directory(
    directory=commands_dataset,
    batch_size=BATCH_SIZE,
    validation_split=TEST_DATASET_SIZE + VALIDATION_DATASET_SIZE,
    seed=0,
    output_sequence_length=SAMPLE_RATE,
    subset='both')
\end{lstlisting}

Učitavanjem smo dobili dva skupa podataka (engl. dataset): skup za treniranje i validacijski skup.
Skup za treniranje koristi se, kao što mu ime kaže, za treniranje neuronske mreže,
dok se validacijskim skupom nakon svake epohe (pojam epoha je objašnjena u poglavlju \ref{}) 
provjerava točnost modela, podešavaju
hiperparamertri i sprječava prenaučenost (te podatke model nije "vidio" tijekom treniranja).
Uz to, od validacijskog skupa se još odvoji jedan dio koji se zove testni skup. On služi
za konačno testiranje točnosti neuronske mreže jer te podatke mreža nije vidjela niti
u jednom trenutku tokom treniranja. Veličine tih skupova određuju parametri 
TEST\_DATASET\_SIZE i VALIDATION\_DATASET\_SIZE koje je također moguće podesiti na početku
bilježnice. 

Nakon što su skupovi podataka učitani, potrebno je generirati značajke za svaki zvučni zapis.
Detaljni opis generiranja značajki objašnjen je u \ref{sec:gen}. U dodatku \ref{} prikazano
je generiranje značajki korišteno za ove podatke (razlika od onog objašnjenog u spomenutom poglavlju
je što se ovo izvodi na osobnom računalu i napisano je u programskom jeziku Python).
Na slici \ref{pic:mfccpython} prikazani su valni oblici nasumičnih zvučnih zapisa određenih 
kategorija te pripadna matrica značajki gerirana na spomenuti način.

\begin{figure}[htb]
    \centering
    \includegraphics[width=1\linewidth]{Chapters/neuronska_mreza/dataset/mfcc.png} 
    \caption{Zvučni signali i pripadna matrica MFC koeficijenata}
    \label{pic:mfccpython}
\end{figure}

U ovom trenutku skupi podataka pripremljeni su za treniranje. Sljedeći korak je definicija 
strukture neuronske mreže.
\section{Struktura modela neuronske mreže}

Principi strukturiranja konvolucijske neuronske mreže za klasifikaciju matričnih podataka
poput upravo pripremljenih prate opis u poglavlju o CNN-ovima \ref{sec:cnn}.
Uz to, postoji zahtjev za što manjim modelom jer je isti potrebno implementirati na 
mikrokontrolerskoj platformi koja ima ograničavajuće memorijske resurse. Također,
vrijeme potrebno buđenje neuronske mreže, tj. kašnjenje koje unosi mreža
implementirana na mikrokontroleru izravno utječe na performanse sustava 
koji bi trebao raditi u stvarnom vremenu. Imajući to na umu, izgradit ćemo
mrežu dovoljno jednostavnu da pokrije spomenute uvjete, a s druge strane dovoljno
složenu, tako da je sposobna pravilno klasificirati ulazne podatke.

Ulazni podaci su matrice dimenzija (32, 41, 12, 1). Podmatrica dimenzija (41, 12) 
predstavlja matricu značajki pojedinog zvučnog zapisa. U konfiguraciji je odabrano 
12 MFC koeficijenata (od 2. do 13.), a zbog veličine prozora (WINDOW\_SIZE) 
koja iznosi 512 (32 ms) i veličine koraka (STEP\_SIZE) koja iznosi 384 (24 ms)
jedna sekunda zapisa se sastoji od 41 vremenskog okvira. Dodatne dimenzije matrice
predstavljaju redom broj takvih matrica koje se odjednom daju mreži na treniranje 
(BATCH\_SIZE) te dimenzija slike koja u ovom slučaju iznosi jedan. Konvolucijske
neuronske mreže također mogu raditi s višekanalnim matricama kao što su RGB slike. U tom 
slučaju svaki kanal predstavlja prisutnost određene boje u slici. Matrice značajki
generirane nad zvučnim zapisima ponašaju se kao crno-bijele slike (svaka vrijednost
predstavlja svjetlinu određenog piksela).

Ulazni sloj u neuronsku mrežu prate dva konvolucijska s pripadnim slojevima za
poduzorkovanje. Prvi konvolucijski sloj ima 32 jezgre veličine \texttt{3x3},
a drugi njih 16 iste veličine. Oba sloja za poduzorkovanje rade s matricom
veličine \texttt{2x2} te pomakom iznosa dva. Slijedi ih podmreža koja se sastoji 
od dva potpuno povezana sloja 
s, redom, 8 i 16 neurona te izlazni sloj koji ima točno 7 neurona (svaki za jednu
klasifikacijsku kategoriju). Aktivacije svih slojeva su "ReLu", dok izlazni sloj
koristi "softmax" aktivaciju. Isječak koda \ref{code:network} prikazuje postupak
izgradnje opisane mreže.

\begin{lstlisting}[language=C++, caption=Struktura mreže, label=code:network]
model = tf.keras.Sequential([
    layers.Input(shape=input_shape),    # Input layer
    layers.Conv2D(32, kernel_size=3, padding='same', activation='relu'),
    layers.MaxPooling2D(pool_size=2, strides=2, padding='same'),
    layers.Conv2D(16, kernel_size=3, padding='same', activation='relu'),
    layers.MaxPooling2D(pool_size=2, strides=2, padding='same'),

    layers.Flatten(),   # Flatten the data for fully connected layers
    layers.Dense(8, activation='relu'),   # Fully connected layer
    layers.Dropout(0.1),                  # Dropout layer with 10% rate
    layers.Dense(16, activation='relu'),  # Fully connected layer
    layers.Dropout(0.1),                  # Dropout layer with 10% rate
    layers.Dense(num_labels, 'softmax'),  # Output layer (softmax)
])
\end{lstlisting}

Na slici \ref{pic:struktura} prikazan je model neuronske mreže s pripadnim brojem
parametara te oblikom podataka između slojeva. Oblik podataka ima prvu dimenziju 
neodređenu (na slici "?") zbog toga što se mreža može trenirati s proizvoljnom 
veličinom grupe (brojem uzoraka koji se odjednom daju mreži).

\begin{figure}[htb]
    \centering
    \includegraphics[width=1\linewidth]{Chapters/neuronska_mreza/struktura/model.png} 
    \caption{Neuronska mreža \cite{netron}}
    \label{pic:struktura}
\end{figure}


\section{Treniranje i vrednovanje modela}
\label{sec:training}

Nakon definiranja strukture modela neuronske mreže, na red je došlo treniranje. Podaci 
su u ovom trenutku podijeljeni u skup za treniranje, validacijski te testni skup. Također,
svaki podatak (zvučni zapis) pretvoren je u dvodimenzijsku matricu značajki veličine 41x12.


\begin{lstlisting}[language=C++, caption=Konfiguracija za treniranje, label=code:modelcompile]
model.compile(
    optimizer=tf.keras.optimizers.Adam(),
    loss=tf.keras.losses.SparseCategoricalCrossentropy(),
    metrics=['accuracy'],
)
\end{lstlisting}

U isječku koda \ref{code:modelcompile} prikazana je priprema modela za treniranje.
Za optimizacijski postupak odabran je Adam algoritam (engl. \textit{Adaptive Moment Estimation}).
Adam algoritam je vrsta gradijentnog spusta koja koristi prilagodljivu stopu učenja.
Za funkciju gubitka odabrana je kategorička unakrsna entropija (engl. \textit{Sparse Categorical
Crossentropy}). Ona se koristi za treniranje višeklasnih klasifikacijskih modela, a
matematički je opisana u nastavku \ref{eq:crossentropyloss}.

\begin{equation}
    \label{eq:crossentropyloss}
    L = - \frac{1}{N} \sum_{i=1}^{N} \log p(y_i)
\end{equation}

gdje:
\begin{itemize}
    \item \( L \) je gubitak.
    \item \( N \) je broj uzoraka.
    \item \( y_i \) je oznaka primjera (klasa).
    \item \( p(y_i) \) vjerojatnost predikcije za ispravnu klasu.
\end{itemize}

Nakon prolaska BATCH\_SIZE (u našem slučaju 32) uzoraka kroz mrežu, računa se
gubitak na opisani način te se ažuriraju težine mreže (gradijentnim spustom).
Prolazak svih uzoraka kroz mrežu označava kraj jedne epohe. Treniranje traje 
proizvoljan broj epoha, a u ovom slučaju može se konfigurirati varijablom EPOCHS
(u našem slučaju 50).
Posljednji parametar kojim je konfigurirana mreža je metrika koja se koristi za
vrednovanje modela. U ovom slučaju koristi se točnost (engl. accuracy) koja
predstavlja postotak točno klasificiranih uzoraka u odnosu na ukupan broj uzoraka.

Početak treniranja modela prikazan je u isječku koda \ref{code:training}. Modelu
su predani testni i validacijski skupovi podataka. Uz to, postavljeni su uvjeti ranijeg
zaustavljanja treniranja (engl. \textit{Early Stopping}) jer se može dogoditi da model
konvergira u minimum funkcije gubitka prije isteka predviđenog broja epoha. Nakon
što model prestane smanjivati funkciju gubitka na validacijskom skupu, treniranje
se smatra završenim, a model se sprema u stanje s najmanjim gubitkom (nije nužno 
stanje nakon posljednje odrađene epohe).

\begin{lstlisting}[language=Python, caption=Trening, label=code:training]
history = model.fit(
    train_mfcc_dataset,
    validation_data=validation_mfcc_dataset,
    epochs=EPOCHS ,
    callbacks=tf.keras.callbacks.EarlyStopping(verbose=1, patience=10, restore_best_weights=True),
)
\end{lstlisting}

U nastavku je prikazan proces treniranja. Vidljivo je kako se funkcija
gubitka smanjuje s vremenom, a točnost modela raste. Model je konvergirao nakon 20-ak
epoha te je uzeo stanje s kraja 18. epohe. U postavkama modela namješteno je da se
treniranje ne zaustavi odmah nego da da modelu još određeni broj epoha koji je
u našem slučaju 10 (patience=10).

{
\tiny
%\scriptsize
\begin{verbatim}
Epoch 1/50
662/662 [==============================] - 3s 5ms/step - loss: 1.3427 - accuracy: 0.4707 - val_loss: 0.8837 - val_accuracy: 0.6935
Epoch 2/50
662/662 [==============================] - 3s 5ms/step - loss: 0.9060 - accuracy: 0.6649 - val_loss: 0.6565 - val_accuracy: 0.7714
Epoch 3/50
662/662 [==============================] - 3s 5ms/step - loss: 0.7439 - accuracy: 0.7320 - val_loss: 0.5586 - val_accuracy: 0.8025
Epoch 4/50
662/662 [==============================] - 3s 5ms/step - loss: 0.6457 - accuracy: 0.7659 - val_loss: 0.4887 - val_accuracy: 0.8230
...
Epoch 28/50
662/662 [==============================] - 3s 5ms/step - loss: 0.2817 - accuracy: 0.9040 - val_loss: 0.3869 - val_accuracy: 0.8823
Restoring model weights from the end of the best epoch: 18.
Epoch 28: early stopping
\end{verbatim}
}

Na slici \ref{pic:accuracy} prikazana su dva grafa. Lijevi prikazuje funkciju gubitka
na skupu za treniranje i validacijskom skupu, dok desni prikazuje točnost modela na istim skupovima.
Obje funkcije prikazane su u ovisnosti o broju odrađenih epoha treniranja. Vidljivo je
kako se funkcija gubitka smanjuje s vremenom, a točnost raste. Nakon 20-ak epoha, funkcije
se stabiliziraju. 

\begin{figure}[htb]
    \centering
    \includegraphics[width=1\linewidth]{Chapters/neuronska_mreza/trening/acc.png} 
    \caption{Gubitak i točnost modela}
    \label{pic:accuracy}
\end{figure}

Konačni iznos funkcije gubitka na testnom skupu iznosi 0.2817, a točnost modela 0.9040, dok
na validacijskom skupu iznosi redom 0.3869 i 0.8823. Međutim, konačna ocjena rezultata
treniranja modela donosi se na temelju testnog skupa. To su podaci koje model nije vidio
niti u jednom trenutku treniranja i predstavljaju podatke kakve će model vidjeti u stvarnom
svijetu. Funkcija gubitka na tom skupu iznosi 0.3480, dok točnost iznosi 0.8814.

Na slici \ref{pic:confmtrx} prikazana je konfuzijska matrica napravljena nad
testnim skupa podataka. Ona prikazuje koliko je puta model pogriješio u klasifikaciji
određene klase. Stupci matrice predstavljaju stvarne klase, a reci predviđene klase
(izlaz treniranog modela). Na dijagonali matrice nalaze se točne klasifikacije, dok
se izvan dijagonale se nalaze pogreške. Vidljivo je kako je model najviše griješio
u klasifikaciji klase "unknown". To je slučaj zbog toga što se u toj klasi nalaze
zvučni zapisi različitih riječi. Drugim riječima, ta klasa je najraznolikija i najteža
za klasificirati te je zbog toga ovakav rezultat očekivan.

\begin{figure}[htb]
    \centering
    \includegraphics[width=0.7\linewidth]{Chapters/neuronska_mreza/trening/image.png} 
    \caption{Konfuzijska matrica}
    \label{pic:confmtrx}
\end{figure}

\section{Usporedba s drugim modelima na istom skupu podataka}

Usporediti naučeni model s postojećim modelima nije jednostavno zato što
se većina modela temelji na složenijim arhitekturama koje imaju veći broj parametara te
uz svoje rezultate ne prilažu način implementacije na mikrokontroleru. Međutim,
u tablici \ref{tab:models} prikazana je usporedba ovog modela s drugima. Uz naš trenirani model,
ubačen je identičan model treniran na samo dvije glasovne naredbe (ukupno četiri klase).
Iz tablice je vidljivo da naš model zauzima najmanje memorije uz sličnu točnost za 4 klase te
nešto manju za 7 klasa. Točnost bi se jednostavno mogla povećati složenijim potpuno povezanim
slojevima na izlazu trenirane mreže, međutim ovo se čini kao najbolji kompromis između veličine
mreže i njene točnosti.

\begin{table}[htb]
    \centering
    \begin{tabular}{|l|r|r|}
        \hline
        \textbf{Model} & \textbf{Accuracy (\%)} & \textbf{Model Size (KB)} \\ \hline
        Naš CNN (7 klasa) & 88.2 & 15.7 \\ 
        Naš CNN (4 klase) & 93.6 & 15.6 \\ 
        DNN (Deep Neural Network)          \cite{zhang2017hello} & 84.6 & 80 \\
        CNN (Convolutional Neural Network) \cite{zhang2017hello} & 91.6 & 79 \\
        LSTM (Long Short-Term Memory)      \cite{zhang2017hello} & 92.9 & 79.5 \\
        CRNN (Convolutional RNN)           \cite{zhang2017hello} & 94.0 & 79.7 \\
        DS-CNN (Depthwise Separable CNN)   \cite{zhang2017hello} & 94.4 & 38.6  \\
        TripletLoss-res15 \cite{triplet} & 95.2 & 237 \\
        BC-ResNet-8 \cite{res} & 98.7 & 520 \\
        WaveFormer \cite{waveformer} & 98.8 & 130  \\
        \hline
    \end{tabular}
    \caption{Veličine različitih oblika modela i procentualna ušteda}
    \label{tab:models}
\end{table}

\section{Prilagodba za implementaciju na mikrokontroleru}
\label{sec:convert}

Veličina pojedinog sloja treniranog modela neuronske mreže prikazana je na slici \ref{pic:size}.
Ukupni broj parametara koje mreža ima iznosi 9439 što u memoriji zauzima malo manje od 
37 KB.

\begin{figure}[htb]
    \centering
    \includegraphics[width=0.65\linewidth]{Chapters/neuronska_mreza/convert/struktura.png} 
    \caption{Veličina trenirane neuronske mreže}
    \label{pic:size}
\end{figure}

Međutim, spremljeni model u memoriji osim vrijednosti parametara mora imqti i informaciju 
o samoj strukturi mreže što znatno povećava sami memorijski otisak. Datoteka s nastavkom
".pb" (engl. protobuff) čuva sve informacije potrebne za korištenje treniranog modela,
a konretni model u tom obliku zauzima nešto više od 173 KB. Korištenje takvog modela
na mikrokontroleru nije prihvatljivo niti zbog veličine niti zbog oblika zapisa. Zbog 
toga je potrebno prilagoditi model. Tensorflow Lite biblioteka omogućava vrlo jednostavnu
promjenu formata spremanja informacije o treniranom modelu. Format s nastavkom ".tflite"
sažima model na nešto više od 41 KB. Međutim, postoji još nešto što je moguće napraviti
kako bismo saželi model još više te ga pretvorili u oblik pogodan za korištenje na
mikrokontroleru. Spomenuta metoda sažimanja zove se kvantizacija, a oblik u kojem će model
biti spremljen zove se "flatbuffer".

Kvantizacija je metoda optimizacije modela kojom se smanjuje broj bitova potrevnih za
spremanje informacije o parametrima modela. Ova redukcija preciznosti (s 32 na 8 bitova)
pridonosi smanjenju veličine modela i ubrzanju izvođenja, a neznatno utječe na točnost
modela \cite{}. Ovim zahvatom veličina modela smanjena je na 16 KB. 

Flatbuffer je oblik za serijalizaciju podataka razvijen u Googleu. Dizajniran je
za učinkovitu pohranu i pristup podacima \cite{}. 
Pretvorbom treniranog modela u ovakav oblik
dobiveno je polje podataka spremno za korištenje programskim jezikom C.

Tablica \ref{tab:model_sizes} prikazuje veličine modela u različitim koracima prilagodbe.
Konačni model prihavtljive je veličine za implementaciju na mikrokontrolerskom sustavu, 
a iznosi svega 9\% početne veličine modela.

\begin{table}[htb]
    \centering
    \begin{tabular}{|l|r|}
        \hline
        \textbf{Model} & \textbf{Veličina(B)} \\ \hline
        Početni model & 173241\\ \hline
        TF Lite & 41644 \\ \hline
        TF Lite + kvantizacija & 15704 \\ \hline
    \end{tabular}
    \caption{Veličine različitih oblika modela}
    \label{tab:model_sizes}
\end{table}