\section{Skup podataka za treniranje}
\label{sec:dataset}

Prvi korak u izgradnji sustava koji koristi bilo kakav tip modela strojnog učenja
je odabir i priprema prikladnog skupa podataka na kojem će se taj model trenirati.
Budući da je cilj sustava prepoznavanje govornih naredbi (eng. keyword spotting),
savršen otvoreni skup podataka je \cite{speechcommandsv2}. Riječ je o skupu koji
se sastoji od oko 105000 zvučnih isječaka duljine oko jedne sekunde (frekvencija
zapisa je 16 kHz) u kojima ljudi
izgovaraju jednu od 35 različitih riječi. Također, skup ima nekoliko vrsta 
pozadinske buke koja je ključna za rad ovakvog sustava u pravom svijetu. 
U prikupljanju podataka sudjelovalo je oko 2600 ljudi iz cijelog svijeta.
U privitku je prikazana tablica koja prikazuje od kojih se riječi skup sastoji
te koliko snimaka pojedinih riječi postoji \ref{tab:word_frequency}.


\section{Priprema podataka}
\label{sec:data}

Nakon odabira skupa na kojem će model biti treniran, potrebno je pripremiti
podatke. Zbog ograničenih resursa na mikrokontrolerskom sustavu, nije moguće
(a niti potrebno za konkretnu namjenu) izgraditi sustav koji će moći
prepoznati sve riječi iz skupa. Potrebno je odabrati samo podskup
za koji će sustav uspješno klasificirati izgovorenu riječ. Cjelokupni
proces pripreme podataka, treniranja i validacije modela popraćen je
Jupyter bilježnicom koja se nalazi u GitHub repozitoriju
\cite{balic_keyword_spotting}.

Projektirani sustav mora moći prepoznati naredbe koje su izgovorene.
Zbog toga, mora biti sposoban odbaciti sve ono što nije naredba koja se
nalazi u odabranom skupu. Iz toga proizlazi da jedna od kategorija (klasa)
na kojoj će model biti treniran jest pozadinska buka. Odabrani skup podataka
sadrži zvučne zapise kao što je zvuk perilice posuđa, zvuk vode koja teče
iz slavine, bijeli šum te ružičasti šum. Bijeli šum (eng. white noise) je vrsta
signala koji u sebi sadrži sve frekvencije i sve imaju isti intenzitet,
dok se ružičasti šum također sastoji od svih frekvencija, ali veći intenzitet
imaju niže frekvencijske komponente \cite{noise}. Međutim, zbog toga što skup
podataka zadrži
svega nekoliko minuta takvih zvučnih zapisa, potrebno je na neki način dopuniti
taj podskup podataka. Naime mreža koju treniramo će preferirati klasu neku od klasa
ako takvih primjera ima mnogo više od primjera ostalih klasa. Drugim riječima,
skup podataka za treniranje mora biti balansiran (primjera iz svake klase treba
biti otprilike jednak broj) \cite{balance}. Uz to, za rad u stvarnom svijetu, nije loše u skup
dodati zvučni zapis snimljen upravo na sustavu koji će i akvizirati podatke iz okoline.
Uz snimke pozadinskih zvukova iz skupa podataka, dodane su snimke snimljenje upravo
na sustavu gdje će sve biti implementirano, a zadrže karakteristične pozadinske
zvukove okruženja u kojem će se nalaziti uređaj. Budući da su svi zvučni zapisi 
riječi u skupu podataka duljine od otprilike jedne sekunde, pozadinske zvukove
je potrebno izrezati na identičnu duljinu kako bi mreža mogla primati vrlo 
precizno definiranu vrstu ulaznih podataka. O broju riječi iz klasa koje su odabrane 
kao naredbe ovisit će koliko trebamo imati isječaka koji će predstavljati
klasu pozadinskih zvukova. Metoda kojoj lako možemo "umnožiti" broj pozadinskih 
zvukova zove se augmentacija.

AUGMENTACIJA...

Druga kategorija mora biti sastavljena od kombinacije različitih riječi za koje
ne želimo klasifikaciju, tj. nisu odabrane u podskup naredbi. Ostatak
kategorija bit će odabrane riječi što znači da će ukupan broj klasifikacijskih
kategorija biti za dva veći od broja odabranih naredbi.

ODABRANE NAREDBE.....

broj zapisa u svakoj klasi....


podjela na trenig/valid(test)




